\documentclass[10pt,a4paper]{article}
\usepackage{tikz} % for drawing figures
\usepackage{amsmath} % for equations
\usepackage{url} % for URLs
\usepackage{graphicx}
\usepackage{hyperref}
\usepackage{pslatex}
\usepackage{apacite}
\usepackage{placeins}
\usepackage{subcaption}
\date{}

\begin{document}

\begin{minipage}{.99\linewidth}
\centering
\large{

\textbf{Supplementary materials} 

Ambiguity classes}
\end{minipage}

\section*{Experiment 1}

Figure~\ref{fig:ambiguity-classes-exp1} shows three exemplar scenarios for three representative ambiguity classes. 
Let us consider the first class in more detail.
In the scenario $S$ on the left side of Figure~\ref{fig:ambiguity-classes-exp1a}, 
the utterance ``blue'' refers either to the blue square or the blue circle.
The picked object, that is, the blue circle, is unique in its shape (circle) and shares the other non-referenced property with both other objects (that is, its plain pattern). 
The referenced but not picked object (that is, the blue square), shares its shape with the non-referenced object. 
In the scenario $S$ in the center, the referenced two red objects differ in texture but share shape with the non-referenced object.
In the scenario $S$ on the right, the referenced two solid objects can be contrasted in their color but share their shape with the third object.



\begin{figure*}[htb]
\centering
\begin{subfigure}{\linewidth}
\begin{subfigure}[b]{0.3\linewidth}
\centering
\begin{tikzpicture}
\node[](o1) at (-1,0){\includegraphics[width=.75cm]{appendix/311.png}};
\node[](o2) at (0,0){\includegraphics[width=.75cm]{appendix/211.png}};
\node[](o3) at (1,0){\includegraphics[width=.75cm]{appendix/313.png}};
\draw[dashed, color=orange,line width=1mm] (o2.south west) rectangle ++(1.cm,1.cm);
\node[] at ([yshift=10pt]o2.north){Utterance: \textbf{``blue"}};
\end{tikzpicture}
\end{subfigure}
\hspace*{0.3cm}
\begin{subfigure}[b]{0.3\linewidth}
\begin{tikzpicture}
\node[](o1) at (-1,0){\includegraphics[width=.75cm]{appendix/212.png}};
\node[](o2) at (0,0){\includegraphics[width=.75cm]{appendix/213.png}};
\node[](o3) at (1,0){\includegraphics[width=.75cm]{appendix/222.png}};

\draw[dashed, color=orange,line width=1mm] (o3.south west) rectangle ++(1.cm,1.cm);

\node[] at ([yshift=10pt]o2.north){Utterance: \textbf{``red"}};
\end{tikzpicture}
\end{subfigure}
\hspace*{0.3cm}
\begin{subfigure}[b]{0.3\linewidth}
\begin{tikzpicture}
\node[](o1) at (-1,0){\includegraphics[width=.75cm]{appendix/112.png}};
\node[](o2) at (0,0){\includegraphics[width=.75cm]{appendix/131.png}};
\node[](o3) at (1,0){\includegraphics[width=.75cm]{appendix/111.png}};
\draw[dashed, color=orange,line width=1mm] (o1.south west) rectangle ++(1.cm,1.cm);
\node[] at ([yshift=10pt]o2.north){Utterance: \textbf{``solid"}};
\end{tikzpicture}
\end{subfigure}
\caption{The utterance references two objects, the picked object has one non-referenced unique feature, while the other, non-referenced feature is shared amongst all three objects.
The other referenced, but not chosen object, shares its other feature with the non-referenced object.} 
\label{fig:ambiguity-classes-exp1a}
%21 32 12
\end{subfigure}
%%%%%%     Subfigure Part 2
%class 21 22 12
\begin{subfigure}{\linewidth}
\centering
\begin{subfigure}[b]{0.3\linewidth}
\centering
\begin{tikzpicture}
\node[](o1) at (-1,0){\includegraphics[width=0.75cm]{appendix/133.png}};
\node[](o2) at (0,0){\includegraphics[width=0.75cm]{appendix/122.png}};
\node[](o3) at (1,0){\includegraphics[width=0.75cm]{appendix/232.png}};

\draw[dashed, color=orange,line width=1mm] (o2.south west) rectangle ++(1.00cm,1.00cm);

\node[] at ([yshift=10pt]o2.north){Utterance: \textbf{``red"}};
\end{tikzpicture}
\end{subfigure}
\hspace*{0.3cm}
\begin{subfigure}[b]{0.3\linewidth}
\begin{tikzpicture}
\node[](o1) at (-1,0){\includegraphics[width=0.75cm]{appendix/133.png}};
\node[](o2) at (0,0){\includegraphics[width=0.75cm]{appendix/122.png}};
\node[](o3) at (1,0){\includegraphics[width=0.75cm]{appendix/232.png}};

\draw[dashed, color=orange,line width=1mm] (o1.south west) rectangle ++(1.00cm,1.00cm);

\node[] at ([yshift=10pt,xshift=0pt]o2.north){Utterance: \textbf{``polka-dotted"}};
\end{tikzpicture}
%		\subcaption{Another combination, here the polka-dotted objects differ in shape and color.}
\end{subfigure}
\hspace*{0.3cm}
\begin{subfigure}[b]{0.3\linewidth}
\begin{tikzpicture}
\node[](o1) at (-1,0){\includegraphics[width=0.75cm]{appendix/212.png}};
\node[](o2) at (0,0){\includegraphics[width=0.75cm]{appendix/221.png}};
\node[](o3) at (1,0){\includegraphics[width=0.75cm]{appendix/311.png}};

\draw[dashed, color=orange,line width=1mm] (o2.south west) rectangle ++(1.00cm,1.00cm);

\node[] at ([yshift=10pt]o2.north){Utterance: \textbf{``circle"}};
\end{tikzpicture}
%	\subcaption{Two circles differ in texture and color.}
\end{subfigure}
%21 22 12
\caption{The utterance $u$ references two objects whereby both objects only share the uttered feature. The third object shares one feature with each of the two referenced objects.}
\end{subfigure}
%%%%%%     Subfigure Part 3	
\begin{subfigure}{\linewidth}
\centering
\begin{subfigure}[b]{0.3\linewidth}
\centering
\begin{tikzpicture}
\node[](o1) at (-1,0){\includegraphics[width=0.75cm]{appendix/131.png}};
\node[](o2) at (0,0){\includegraphics[width=0.75cm]{appendix/221.png}};
\node[](o3) at (1,0){\includegraphics[width=0.75cm]{appendix/231.png}};

\draw[dashed, color=orange,line width=1mm] (o1.south west) rectangle ++(1.00cm,1.00cm);

\node[] at ([yshift=10pt]o2.north){Utterance: \textbf{``blue"}};
\end{tikzpicture}
%		\subcaption{The objects can differ in shape, texture, and color. The utterance applies to all 3 objects but they differ in shape and texture. Solid objects or squares are absent here, so we cannot learn whether the listener likes squares or solid objects.}
\end{subfigure}
\hspace{0.3cm}
\begin{subfigure}[b]{0.3\linewidth}
\begin{tikzpicture}
\node[](o1) at (-1,0){\includegraphics[width=0.75cm]{appendix/123.png}};
\node[](o2) at (0,0){\includegraphics[width=0.75cm]{appendix/122.png}};
\node[](o3) at (1,0){\includegraphics[width=0.75cm]{appendix/112.png}};

\draw[dashed, color=orange,line width=1mm] (o3.south west) rectangle ++(1.00cm,1.00cm);

\node[] at ([yshift=10pt]o2.north){Utterance: \textbf{``cloud"}};
\end{tikzpicture}
%		\subcaption{Another combination, here all objects are clouds but the picked object shares being red with another object.}
\end{subfigure}
\hspace*{0.3cm}
\begin{subfigure}[b]{0.3\linewidth}
\begin{tikzpicture}
\node[](o1) at (-1,0){\includegraphics[width=0.75cm]{appendix/311.png}};
\node[](o2) at (0,0){\includegraphics[width=0.75cm]{appendix/211.png}};
\node[](o3) at (1,0){\includegraphics[width=0.75cm]{appendix/313.png}};

\draw[dashed, color=orange,line width=1mm] (o2.south west) rectangle ++(1.00cm,1.00cm);

\node[] at ([yshift=10pt]o2.north){Utterance: \textbf{"solid"}};
\end{tikzpicture}
\end{subfigure}
\caption{In this third exemplar ambiguity class, the utterance refers to all three objects. The picked object shares one feature with one other object and has one feature just for itself while the other two objects share it.}
%32 21 12
\end{subfigure}
\caption{Three exemplar scenarios $S$, constraining utterance $u$, and chosen object $s$ are shown for three exemplar ambiguity classes for Experiment 1.}
\label{fig:ambiguity-classes-exp1}
\end{figure*}

\section*{Experiment 2}

Figure~\ref{fig:ambiguity-classes-exp2} shows three exemplar scenarios for three representative  ambiguity classes. 
Let us again consider the first class in more detail.
In the scenario $S$ on the left side of Figure~\ref{fig:ambiguity-classes-exp2a}, 
all three objects share the feature pattern (solid), while two share the color (blue), and the other two share the shape (square).
As a result, uttering \textit{green} or \textit{circle} will give no choice to the listener because the utterance identifies one unique object. 
On the other hand, uttering \textit{solid} will let the listener choose freely, while uttering \textit{blue} or \textit{square} will give a specific choice between two objects, that is, between the blue circle and the blue square or between the blue square or the green square, respectively..
In the scenario $S$ in the center, the objects share the shape (circle), two share the pattern (solid), and the other two share the color (red).
Here, \textit{circle} references all three objects, \textit{red} or \textit{solid} reference pairs of objects, and \textit{striped} or \textit{green} reference one unique object each.
In the scenario $S$ on the right, the object again share the shape (cloud), two share the pattern (solid), while the other two share the color (blue).


\begin{figure*}[!htb]
\centering
\begin{subfigure}{\linewidth}
\begin{subfigure}[t]{0.3\linewidth}
\centering
\begin{tikzpicture}
\node[](o1) at (-1,0){\includegraphics[width=0.75cm]{appendix/311.png}};
\node[](o2) at (0,0){\includegraphics[width=0.75cm]{appendix/211.png}};
\node[](o3) at (1,0){\includegraphics[width=0.75cm]{appendix/313.png}};

\end{tikzpicture}
%		\subcaption{}
\end{subfigure}
\hspace*{0.3cm}
\begin{subfigure}[t]{0.3\linewidth}
\begin{tikzpicture}
\node[](o1) at (-1,0){\includegraphics[width=0.75cm]{appendix/212.png}};
\node[](o2) at (0,0){\includegraphics[width=0.75cm]{appendix/213.png}};
\node[](o3) at (1,0){\includegraphics[width=0.75cm]{appendix/222.png}};

\end{tikzpicture}
%		\subcaption{Another combination, h}
\end{subfigure}
\hspace*{0.3cm}
\begin{subfigure}[t]{0.3\linewidth}
\begin{tikzpicture}
\node[](o1) at (-1,0){\includegraphics[width=0.75cm]{appendix/131.png}};
\node[](o2) at (0,0){\includegraphics[width=0.75cm]{appendix/112.png}};
\node[](o3) at (1,0){\includegraphics[width=0.75cm]{appendix/111.png}};

\end{tikzpicture}
%		\subcaption{Another combination following the same principle.}
\end{subfigure}
\caption{In this exemplar ambiguity class, one feature is shared by all three objects, while the two other features allow the distinction between two different pairs of objects and the reference of one of two uniquely identifiable objects.}
\label{fig:ambiguity-classes-exp2a}
%322b
\end{subfigure}
%%%%%%%%%%%%%%%%%%% start of second part 
\begin{subfigure}{\linewidth}
\centering
\begin{subfigure}[t]{0.3\linewidth}
\centering
\begin{tikzpicture}
\node[](o1) at (-1,0){\includegraphics[width=0.75cm]{appendix/133.png}};
\node[](o2) at (0,0){\includegraphics[width=0.75cm]{appendix/122.png}};
\node[](o3) at (1,0){\includegraphics[width=0.75cm]{appendix/232.png}};		
\end{tikzpicture}
%		\subcaption{Uttering \textit{cloud}, \textit{red} or \textit{polka-dotted} will each pick out a different pair that lets us learn about a combination of two preferences. If we utter \textit{cloud} for example, a preference for green or red and a preference for polka-dotted or striped may figure in the listeners choice. Uttering \textit{circle}, \textit{green} or \textit{striped} will pick out a single object, giving the listener no choice.}
\end{subfigure}	
\hspace*{0.3cm}
\begin{subfigure}[t]{0.3\linewidth}
\begin{tikzpicture}
\node[](o1) at (-1,0){\includegraphics[width=0.75cm]{appendix/223.png}};
\node[](o2) at (0,0){\includegraphics[width=0.75cm]{appendix/322.png}};
\node[](o3) at (1,0){\includegraphics[width=0.75cm]{appendix/313.png}};
\end{tikzpicture}
%\subcaption{Another combination, here \textit{green}, \textit{striped} or \textit{square} will each pick out a pair. \textit{Solid}, \textit{red} or \textit{circle} will each pick out a single object.}
\end{subfigure}
\hspace*{0.3cm}
\begin{subfigure}[t]{0.3\linewidth}
\begin{tikzpicture}
\node[](o1) at (-1,0){\includegraphics[width=0.75cm]{appendix/212.png}};
\node[](o2) at (0,0){\includegraphics[width=0.75cm]{appendix/221.png}};
\node[](o3) at (1,0){\includegraphics[width=0.75cm]{appendix/311.png}};
\end{tikzpicture}
%		\subcaption{Another combination following the same principle.}
\end{subfigure}
%22b2b
\caption{In this second exemplar ambiguity class, all three feature types allow the identification of pairs of objects or unique objects, where all three features contain one unique feature type, each. As a result, there are three utterances that each pick out a different pair of objects and three other utterances that each reference one single object -- effectively allowing the unique identification of each object as well as the identification of all three possible pairs.}
\end{subfigure}
%%%%%%%%%%%%%  start of third part of figure
\begin{subfigure}{\linewidth}
\centering
\begin{subfigure}[b]{0.3\linewidth}
\centering
\begin{tikzpicture}
\node[](o1) at (-1,0){\includegraphics[width=0.75cm]{appendix/131.png}};
\node[](o2) at (0,0){\includegraphics[width=0.75cm]{appendix/213.png}};
\node[](o3) at (1,0){\includegraphics[width=0.75cm]{appendix/222.png}};		
\end{tikzpicture}
%		\subcaption{Uttering \textit{circle} picks out the green solid one and the red striped one, letting us learn about a combination of preferences. Uttering \textit{solid}, \textit{striped}, \textit{polka-dotted}, \textit{cloud}, \textit{green}, \textit{red} or \textit{blue} would always pick just one object. That way, the listener would have no choice and we could not learn anything.}
\end{subfigure}
\hspace*{0.3cm}
\begin{subfigure}[t]{0.3\linewidth}
\begin{tikzpicture}
\node[](o1) at (-1,0){\includegraphics[width=0.75cm]{appendix/233.png}};
\node[](o2) at (0,0){\includegraphics[width=0.75cm]{appendix/122.png}};
\node[](o3) at (1,0){\includegraphics[width=0.75cm]{appendix/312.png}};		
\end{tikzpicture}
%		\subcaption{Another combination, here all uttering \textit{red} is the only way to let the listener choose.}
\end{subfigure}
\hspace*{0.3cm}
\begin{subfigure}[t]{0.3\linewidth}
\begin{tikzpicture}
\node[](o1) at (-1,0){\includegraphics[width=0.75cm]{appendix/331.png}};
\node[](o2) at (0,0){\includegraphics[width=0.75cm]{appendix/212.png}};
\node[](o3) at (1,0){\includegraphics[width=0.75cm]{appendix/113.png}};	
\end{tikzpicture}
%		\subcaption{Another combination following the same principle.}
\end{subfigure}
\caption{In this third exemplar ambiguity class, two features have three unique values, while one feature allows the identification of a pair of objects.}
\end{subfigure}
%211
\caption{Three exemplar scenarios $S$ are shown for three exemplar ambiguity classes for Experiment 2.} 
\label{fig:ambiguity-classes-exp2}
\end{figure*}

\end{document}